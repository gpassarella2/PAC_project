\usepackage[T1]{fontenc}
\usepackage[utf8]{inputenc}
\usepackage[italian]{babel}
\usepackage[nottoc]{tocbibind}
\usepackage{geometry}

\geometry{
    a4paper,
    left=3cm,
    right=3cm,
    top=3cm,
    bottom=3cm,
    heightrounded
}

\usepackage{setspace}               % -> Interlinea
\onehalfspacing
\setcounter{secnumdepth}{3}         % -> Numerazione indice
\setcounter{tocdepth}{3} 

% --- GRAFICA E COLORI ---
\usepackage{xcolor}                 % IMPORTANTE: Caricato prima di listings
\usepackage{graphicx}               % -> Immagini
\usepackage{tikz}
\usepackage{float}
\usepackage{caption}
\usepackage{subcaption}             % <--- SOSTITUITO subfigure (più moderno)
\usepackage{svg}
\usepackage{pgfplots}               % -> Plot in latex
\pgfplotsset{compat=1.18} 

% --- MATH & FONTS (Ordine Ottimizzato) ---
\usepackage{amsfonts}               % -> Math standard
\usepackage{amsmath}                % -> Math avanzato
\usepackage{newtxtext,newtxmath}    % -> Times New Roman (caricato DOPO amsmath)
\usepackage{xfrac}                  % -> sfrac
\usepackage{rsfso}                  % -> Font per Laplace L
\usepackage{siunitx}                % -> Unità di misura (SI)

% --- CODICE E LISTINGS ---
\usepackage{listings}

% Alias per C++
\lstdefinelanguage{cpp}{
    language=C++
}

\lstset{
    basicstyle=\ttfamily,
    keywordstyle=\color{blue},
    stringstyle=\color{red},
    commentstyle=\color{gray},
    numbers=left,
    numberstyle=\tiny,
    captionpos=b,
    breaklines=true
}

% --- ALTRI PACCHETTI ---
\usepackage{lipsum}                 % -> Lorem ipsum
\usepackage{afterpage}              % -> Blank numbered page
\usepackage{enumerate}
\usepackage[RPvoltages]{circuitikz} % -> Schematic design
\usepackage{fancyvrb}               % -> Codice colorato in verbatim
\usepackage{pdfpages}               % -> Inserisce pdf 
\usepackage{subfiles}
\usepackage{booktabs}               % -> Tabelle belle
\usepackage{makecell}
\usepackage{multirow}
\usepackage[toc,page]{appendix}     % -> Appendici

\addto\captionsitalian{%
   \renewcommand{\appendixtocname}{Appendici}%
   \renewcommand{\appendixpagename}{Appendici}%
}

% --- HYPERREF (Va caricato quasi sempre per ultimo) ---
\usepackage{hyperref}

\hypersetup{
    colorlinks=true,
    linkcolor=black,
    filecolor=black,      
    urlcolor=black,
    pdfborder={1 0 0},
    linkbordercolor=black,
}

% Configurazione autoref italiano
\AtBeginDocument{%
    \renewcommand{\chapterautorefname}{Capitolo}
    \renewcommand{\sectionautorefname}{Sezione}
    \renewcommand{\subsectionautorefname}{Sottosezione}
    \renewcommand{\subsubsectionautorefname}{Sotto-sottosezione}
    \renewcommand{\figureautorefname}{Figura}
    \renewcommand{\tableautorefname}{Tabella}
}

% --- HEADER E FOOTER ---
\usepackage{fancyhdr}               
\pagestyle{fancy}
\fancyfoot[CO, CE]{\thepage}
\renewcommand{\headrulewidth}{0.4pt}
\renewcommand{\footrulewidth}{0.4pt}
\setlength{\headheight}{15pt}

% Roman number helper
\makeatletter
\newcommand*{\rom}[1]{\expandafter\@slowromancap\romannumeral #1@}
\makeatother

% Blank page helper
\newcommand\blankpage{%
    \null
    \thispagestyle{empty}%
    \addtocounter{page}{-1}%
    \newpage
}